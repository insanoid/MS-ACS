
\documentclass[12pt]{article}
\usepackage{geometry}
\geometry{a4paper}
\usepackage{chicago}
\usepackage{tabularx}
\usepackage{booktabs}
\usepackage{multirow}

\title{Collaboration in computer science publications: A study of collaboration rates and citation rates}
\author{Karthikeya Udupa K M (1393456)}
\date{\today}

\begin{document}
\maketitle


\begin{abstract}

\noindent
Collaboration has been considered to be one of the key aspect of in the research field. In the last few decades we have seen people collaborating with each other throughout the world constantly which raises the question about how collaboration effects the research work. It is known that interaction with fellow researchers provides a new insight on the subject hence helping create more effective research. Our research tries to understand how collaboration  effects the quality of the output that is produced by the research and also try to analyse if there has been any change in the way how people collaborate with each other with respect to the field of computer science over the past few years. We achieve this my analysing three journals in computer science by various publishers to gather further information over a period of six years related to both collaboration and citation of the research articles to search for patter and to try and form a relation between the various aspects of research collaboration and understand the role collaboration plays in helping make the research better.




\end{abstract}


\noindent
Over the years, there has been a drastic change in the ways in which researches are being conducted throughout the world. The size of teams have expanded and shifted from the stage where most of the researches were conducted by individual scientists and inventors. Collaborative work is no longer restricted to the members of a single organisation but has expanded to include individuals from multiple organisations separated by great geographical distances. Collaboration has become quite common and is at a rise at an increasingly rapid rate \cite{adams2005scientific}. This can be attributed to the improvements that have been made in the communication technology which were not available to the earlier generations. Additionally, domains are becoming increasingly specialised making collaboration even more important. \cite{sonnenwald2007scientific} defines collaboration as \textit{``interaction taking place within a social context among two or more scientists that facilitates the sharing of meaning and completion of tasks with respect to a mutually shared, superordinate goal''}. It has always been considered a very vital aspect of research, it enables one to share the available resources and the expenses of the research between people and organisations \cite{figg2006scientific}.

\section{Method}
The research articles that were published in the year 2004 and 2010 were collected from three computer science based journals, \textit{Journal of the ACM (JACM)}\footnote{http://jacm.acm.org/}, \textit{Computer Languages, Systems \& Structures (CLSS)}\footnote{http://www.journals.elsevier.com/computer-languages-systems-and-structures/} and \textit{Journal of Logic and Computation (JLC)}\footnote{http://logcom.oxfordjournals.org/}. The choice of these journals were made as they belong to different publishers and cover different aspects of the domain. JACM is published by ACM, CLSS by Elsevier and JLC by Oxford University Press. For each of the journals, all the issues published in the above mentioned years were collected, each constituting of multiple research article. We determined the number of authors who collaborated in the research and the nature of their collaboration, whether it was a local (within the university), national (within the nation) or an international one, however, any paper authored by a single person was not considered for demographical evaluation. For each of the research articles we determined the total count of citations that it had received over the years.

\section{Statistical Analysis}
The purpose of the study is to analyse the relation between collaboration and the impact that it creates on the community, which measured by the count of its citation in other papers. The study also tries to analyse if there is any change in the rate of collaboration over the years and the impact that it has caused. We first look into the each of the journals individually and then compare it with the one published six years.

\subsection{Journal of the ACM (JACM)}
Published by Association for \textit{Computing Machinery (ACM)}, it covers researches that have done in the field of core computer science and has an lasting impact on the field. It is a very old and popular journal, with its first publishing dating back to 1954. For the purpose of our study we are concerned with two publication years, 2004 and 2010. The journal is bimonthly hence publishing a total of six issues a year. 

In 2004, JACM had a total of 31 research articles published in its six issues. Of which only 6 (19.3\%) were individual works. Most of the articles were done by two researchers, constituting of 51.6\% (16 papers) of the overall articles followed by 16 papers by a team of three researchers. Additionally, there was an international collaboration of a group of six researchers which was also published. The papers published by groups of three researchers had the most impact on the community with an average citation of 401.25 per paper which is considerably more then by teams with two researcher (134.4 citations per paper) and also by individual (98.5 citations per paper). The demographics of the collaboration was evenly distributed, a majority of the paper (40\%) being a result of local collaboration and the rest shared between international (32\%) and national collaboration (28.2\%), however, the internationally collaborated papers were much better received with an average citation of 303.1 per paper followed by nationally collaborated ones (197.5 citations per paper). Local collaborations were the least impactful and had an average of 159.4 citation per paper.

In 2010, the number of articles that were published in the journal was reduced to 21. It also saw a reduction in the number of individual articles, to a mere 3 (14.2\%). 30\% and 38\% of the articles were by two and three collaborators respectively. There were 3 articles by more then 3 collaborators as well. The distribution of the impact that the articles also changed over the years, on an average each article received 46.5 citations. The distribution of the citation based on the collaborator count was uneven with individual articles receiving only 22.6 citations per paper on an average whereas two or three author articles had 56 and 42.6 citations on an average. But the highest average was of articles with more then three authors (58.6 citations/week), interestingly all those articles also were internationally collaborated. Overall as well, majority of the papers were internationally collaborated (50\%) followed by local collaboration (38.8\%) and lastly national, which saw a fall to a mere 11.1\%.


Overall, over the years JACM has seen some reduction in its popularity with reduced number of articles, this might be attributed to the emerging of several more field specific journals as well. The reduction is not just in the number of articles, but the average citation per article dropped considerably as well, from 193.3 in 2004 to 46.5 citations. Collaboration however has improved with rise in both the number of collaborators and in the international collaboration. based on the citation that were received, we can also see the impact created by internationally collaborated articles is more then other types of collaborations with respects to JACM.


\subsection{Computer Languages, Systems \& Structures (CLSS)}
Computer Languages, Systems \& Structures is a journal published by Elsevier publication, Netherlands. It primary focus is on research related to programming languages both practical and theoretical, their design, implementation and use. It is a quarterly published journal, amounting to four issues a year. 

In 2004, the journal published a total of 11 articles, 3 of which were by single authors, 10 were by two authors and just one by three collaborators. Two of the articles were written by more then three authors. In that year the journal had an average of 21.6 citations per article. Articles by two authors were the most impactful among others with an average of 25.6 citations. The collaborations were of national (2 papers), international (3 papers) and local nature (3 papers) with citation average of 27, 25.3 and 33 citations per paper respectively.

In 2010, there were a total of 16 papers published in their 4 issues. A majority of the papers were by three collaborator groups (7 papers) followed by individual papers (4 papers). There was a significant reduction in the papers by two authors to just 3 papers and 2 papers by groups of four. The citation rate per paper for the year was reduced to 6.5 of which papers with three collaborators had the highest average of 7.1 citations. 8 of the papers were locally collaborated whereas there were 2 papers each of national and international collaboration, however, the average citation of international ones, 12.8 citations per paper, was much more then that of local ones which was a mere 4.8.

A comparison of the two years indicates a rise in collaboration, especially the ones by three collaborator groups. However, the overall citation rate of the articles was significantly low when compared to its 2004 counterpart. Even though there was a significant rise in the papers by local collaborators, the citation rate of internationally collaborated paper still remains higher, indicating that there is certainly some improvement in the quality of research due when it involves international teams.

\subsection{Journal of Logic and Computation (JLC)}
Journal of Logic and Computation as the name suggests focuses on logic and computing. Published since 1990 by Oxford University Press, it is a popular bimonthly journal. 

In its 6 issues in 2004, there were 33 original research papers published of which 18 (54.5\%) were individual works. 6 and 3 papers were by groups of two and three collaborators respectively in addition to 6 papers by group of 4 collaborators. The average citation rate of the journal in 2004 was 25.5 citations per paper, however the average of papers by groups of 4 was much higher then the average at 44.33 citations/paper. Individual papers were the ones to be cited the least at an average of 20.16 citations per paper. Since the majority of the articles were individual there were only 15 articles for demographic analysis, of which 8 were local, 4 national and 3 internationally collaborated. However, articles from national collaborations were received well and had an average of 90.3 citations per paper which is very high compared to both international (16.7) and local (19.3) and also the average citation rate.

In 2010 there were 48 articles published by the journal. 23 of the articles were authored by two researchers, followed by 13 by individual authors and another 7 by groups consisting of three people. Additionally, 5 of the articles published were by group of more then 3 collaborators. Papers with three collaborators had an average of 27.7 citations per paper which was more then the average rate of 15 citations/paper. It was followed by papers with more then three collaborators at 19.2. Although, paper with two authors had the least citation rate of 11.4, the individual papers with an average of 12.8 had papers which had no citations at all. There were considerably more international collaborations (16 papers) then national (6 papers) or local (13 papers) ones and were cited more often with an average rate of 21.5 citations/paper. The local collaborations were cited the least at 8.4 citations per paper followed by national collaborations (16.6).


\subsection{Conclusion}


\begin{table}[htdp]
\begin{tabular*}{\linewidth}{@{\extracolsep{\fill}}lllllllllllllllll@{}}

  \toprule
  \midrule
  \textbf{Journal} & \textbf{Year (Papers)} &\multicolumn{4}{c}{\textbf{Avg. by Author}} & & \multicolumn{3}{c}{\textbf{Avg. by Collaboration}} \\
  \midrule \addlinespace
  && \textbf{1} & \textbf{2} & \textbf{3} & \textbf{\textgreater 3} & & \textbf{Local} & \textbf{Nat.} & \textbf{Int.} & \\
    \midrule
  \multirow{2}{*}{JACM} & 2004 (31) & 98.5 & 134.4 & 401.2 & 41 && 159.4 &  197.57 & 303.1 \\
  & 2010 (21) & 22.6 & 56 & 42.6 & 58.6 && 23.4 & 98.5 & 60.8\\
  \midrule
  \multirow{3}{*}{CLSS} & 2004 (11) & 13 & 25.6 & 16 & 27.5 && 23 &  30.5 & 25.3 \\
  & 2010 (16) & 5.2 & 2.3 & 7.14 & 13 && 4.8 & 6 & 12.3\\
  \midrule
  \multirow{3}{*}{JLC} & 2004 (33) & 20.1 & 22 & 27 & 44.3 && 19.3 &  90.3 & 16.7 \\
  & 2010 (48) & 12.8 & 11.4 & 27.7 & 19.2 && 8.4 & 16.6 & 21.5\\
  \bottomrule

\end{tabular*}
\caption{Average Citation rates of various journals over the years based on the number of authors and by the nature of the collaboration.}
\label{citation comparison}
\end{table}

\noindent
From the data above, we can conclude several key factors about the relation between collaboration and the citation rate. Throughout the three journals, it was observed that the citation rates of articles with international and national collaboration were far more then those of locally collaborated ones, indicating that people from various geographic regions working together produce better quality results. This maybe attributed to diversity in the nature of their study and thought process which might be lacking when the people working are from the same university. Also collaborated work (two or more authors) had comparatively more citations then individual research work, this is mostly true in all the journals and in also for both the years, indicating multiple people working on the research tend to produce more effective results. This can be attributed to the fresh thought process and the new ideas that are brought together by multiple people working together which cannot be achieved alone. 

\begin{table}[htdp]
\begin{tabular*}{\linewidth}{@{\extracolsep{\fill}}lllllllllllllllll@{}}

  \toprule
  \midrule
  \textbf{Journal} & \textbf{Year (Papers)} &\multicolumn{4}{c}{\textbf{\% by Author}} & & \multicolumn{3}{c}{\textbf{\% by Collaboration}} \\
  \midrule \addlinespace
  && \textbf{1} & \textbf{2} & \textbf{3} & \textbf{\textgreater 3} & & \textbf{Local} & \textbf{Nat.} & \textbf{Int.} & \\
    \midrule
  \multirow{2}{*}{JACM} & 2004 (31) & 19.3 & 51.6 & 25.8 & 3.2 && 40 &  28.2 & 32 \\
  & 2010 (21) & 14.2 & 33 & 38 & 14.2 && 38.8 & 11.1 & 50\\
  \midrule
  \multirow{3}{*}{CLSS} & 2004 (11) & 27.2 & 36.3 & 9.0 & 18.1 && 37.5 & 25 & 37.5 \\
  & 2010 (16) & 25 & 18.7 & 43.7 & 12.5 && 58.3 & 16.6 & 25\\
  \midrule
  \multirow{3}{*}{JLC} & 2004 (33) & 54.5 & 18.1 & 9 & 18.1 && 53.3 &  20 & 26.6 \\
  & 2010 (48) & 27 & 47.9 & 14.5 & 10.4 && 17.1 & 45.7 & 21.5\\
  \bottomrule

\end{tabular*}
\caption{Percentage of papers classified by number of authors and type of collaboration.}
\label{percentage comparison}
\end{table}

Additionally, it was also observed that the trends of individual research also seem to be changing as all three of the journals show an decrease in the number of individual researchers over the years and a substantial increase of collaboration. Similarly, over the duration of six years, 2 out of three journals have shown fall in the percentage of local collaboration, this can be related to relation between national/international collaboration and citation rate mentioned above. Hence we can conclude that collaboration is becoming increasingly common in the journals, leading to better citation rates for the research being done, hence showcasing the increase in the quality of the research. Also the number of national and international collaboration have risen in the duration of the study which too provides a far more effective research paper then the ones which are collaborated locally, however, the sample set is too small to justify our findings entirely and a much more broader sample set needs with more journals from various publishers and with yearly data would be required to analysed and understand the trends more further and clear.



\bibliographystyle{chicago}
\bibliography{Assigment_I}
\end{document}