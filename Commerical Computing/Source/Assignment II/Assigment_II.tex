
\documentclass[12pt]{article}
\usepackage{geometry} % see geometry.pdf on how to lay out the page. There's lots.
\geometry{a4paper} % or letter or a5paper or ... etc
% \geometry{landscape} % rotated page geometry
\usepackage{chicago}

\title{Critically analyse the evidence supporting one or more proposals for how to teach recursion to school children.}
\author{Karthikeya Udupa K M (1393456)}
\date{\today}
%%% BEGIN DOCUMENT
\begin{document}
\maketitle
\noindent

Recursion is an intriguing concept associated with both computer science and mathematics. Originating from the word �recur� which means to return to a place or state, it can be described as a procedure containing a version of itself as a sub-procedure in its definition \cite{leron1986computational}. The fact that it is considered challenging \cite{levy2000recursively} can be attributed to the various complexities involved \cite{leron1988makes} and is hence is dealt with scepticism by students \cite{roberts1986thinking}. Although taught in the initial university years, it has been considered that an early exposure would result in a better understanding of the concept.

Although teaching recursion at school level can certainly have a positive impact but is a challenging task and an entirely differnt teaching methodology would have to be followed to facilitate the children to grasp the concept effectively. A study conducted by K. Gunion et. al. \cite{gunion2009curing} with a group of school children aimed at helping them to recognise and understand recursion in addition to having an encouraging attitude towards recursion. The study constituted of a tactile learning approach paired with programming implementation. They were introduced to the concept of visual recursion with the help of handouts of recursive images, helping them to learn distinguish the concept of recursion along with complex problems which could be broken down and solved with recession such as the Sierpinski�s Carpet and was concluded with tower of hanoi. Through a post-test review it was observed that majority of the students were now able to identify recursive elements in problems and demonstrated basic understanding of recursiveness especially with regards to images. Additionally they were able to apply sorting techniques such as quicksort and divide-and-conquer without any prior instructions. The children now felt considerably comfortable with recursion and were able to describe it in visual terms hence highlighting the significance of visuals in the teaching process \cite{gunion2009paradigm}. Another study by J. Tessler et. al.\cite{tessler2013using} employed a game, Cargo Bot, to demonstrate recursion. The students were initially given an introduction to the concept through visual examples based on the game with tests conducted at various stages of the experiment to analyse their understanding. It was observed that most of the students understood recursiveness better and were more open towards it. They also found the experience fun and were voluntarily exploring further.

Exposing school children to the concept of recursion might be an effective way to not only create a positive impact on them about recursion and computer science but also to provide a foundation for the field. However the teaching technique that is to be used attuned to their needs with ample interactivity, group work and an ability to keep them intrigued. Hence a teaching approach which is both contextual and enjoyable and makes use of visuals can certainly prove a effective way in teaching kids not only recursion but other core concepts of computer science.

\bibliographystyle{chicago}
\bibliography{Assigment_II}
\end{document}