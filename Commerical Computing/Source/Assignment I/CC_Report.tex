
\documentclass[12pt]{article}
\usepackage{geometry} % see geometry.pdf on how to lay out the page. There's lots.
\geometry{a4paper} % or letter or a5paper or ... etc
% \geometry{landscape} % rotated page geometry
\usepackage{chicago}
\usepackage [autostyle]{csquotes}

\title{``Privacy is over. Get used to it'' is often attributed to Scott McNealy, former CEO of Sun. Is he right?}
\author{Karthikeya Udupa K M (1393456)}
\date{\today}
%%% BEGIN DOCUMENT
\begin{document}
\maketitle
\noindent
Privacy is defined as ``A state in which one is not observed or disturbed by other people'' by the Oxford dictionary and in a general scenario it usually relates to the autonomy and secrecy of the user. In reference to the web, privacy usually refers to personal information and the invasion of privacy is usually interpreted as the unauthorised collection, disclosure, or other use of personal information as a direct result of electronic commerce transactions \cite{WANG:1998:CPC}. Since the last two decades we have seamlessly connected the world through the world throughout world wide web but in the process have opened up multiple ways of privacy evasion. We have made efforts to building secure and robust systems, but massive security and privacy failures that occur time and again \cite{TYL:2012:SB}, have proved that privacy might just be a paradox.

The privacy paradoxical state  can partially be attributed to the fact that the web has become so immense, that the user tends to use products and services that accumulate his personal information, the company providing these services might have a relaxed approach towards the information's security, giving an opportunity for privacy failures waiting happen \cite{MM:1990:YHNO}. This also provides the authorities a medium monitor the user's actions and communications which \cite{CLRK:1988:CPC} defines as ``dataveillance''. The rise in the usage of social networks, especially among teenagers, wherein they disclose accurate personal information online \cite{ACQ:2006:FB} has been a primary privacy concern \cite{BRNS:2006:APP,young2009information}. Emergence of ubiquitous, physiological and location aware systems have taken privacy concern to a different dimension, not only does the user have to be concerned with data security but now information such as his location \cite{ting2009}, medical conditions, which in some cases he might not be aware of \cite{fairclough2009fundamentals}, etc may be compromised. The problem, it turns out might not just be developing a secure system and user keeping a vigil on his information, there are various other factors such as market dynamics and political influences that govern the privacy standards \cite{anderson2001information}. Also, not everyone shares the same views on privacy, certain authorities have been making even further attempts to curb the user's privacy \cite{CISPA,PIPA}.

Security techniques are being constantly upgraded to handle the information more securely.  Law's have been passed by government authorities in many parts of the world \cite{oced,ukdp} to manage and regulate how the information being collected from the user and has been applied to public and in some cases private sectors although on the other hand privacy of the user is being compromised for security reasons. Although, amends have constantly been made to ensure upmost standards, privacy at present is a paradox, it is the price you pay if you are to consume services and use products today and at the end it falls on the user to take weighted decisions when dealing with critical personal information.

\bibliographystyle{chicago}
\bibliography{CC_Report}
\end{document}